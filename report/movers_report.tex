\documentclass[a4paper, 11pt]{article}
\usepackage[top=3cm, bottom=3cm, left = 2cm, right = 2cm]{geometry} 
\geometry{a4paper} 
\usepackage[utf8]{inputenc}
\usepackage{textcomp}
\usepackage{graphicx} 
\usepackage{amsmath,amssymb}  
\usepackage{bm}  
\usepackage[pdftex,bookmarks,colorlinks,breaklinks]{hyperref}  
\hypersetup{linkcolor=black,citecolor=black,filecolor=black,urlcolor=black} % black links, for printed output
\usepackage{memhfixc} 
\usepackage{pdfsync}  
\usepackage{fancyhdr}
\pagestyle{fancy}
\setlength{\headheight}{13.6pt}

\title{Movers \\[1ex] \large Theory of Computation, Spring 2024}
\author{Luca Di Bello, Agnese Zamboni, Dimitrios Pagonis, Georgy Batyrev} 
\date{\today}

\begin{document}
\maketitle
\tableofcontents

\section{Problem Description}
\documentclass{arcticle}
\begin{document}

In the \textit{movers satisfiability problem}, a moving company is tasked with relocating all furniture from a building with multiple floors. The objective is to move all furniture to the ground floor within a given time frame (maximum number of time steps). For this task, the company has a team of movers of size $m$, and each mover is identified with a unique name, and can move up or down one floor at a time.

The building has $n$ floors, each identified by a unique integer number. The building contains a set of furniture $F = {f1,f2,...,fn}$ to be moved. Each piece of furniture is located within the floors of the building, and there could be more than one piece of furniture on the same floor. The movers are initially located on the ground floor of the building, and they must move all furniture to the ground floor within a given time frame. By the end of the time frame, all movers and all furniture must be located on the ground floor in order to solve the problem.

When a mover is on the same floor as a piece of furniture, and decides to carry it, the mover and the furniture in question are moved together to the floor below.

\end{document}

\pagebreak

\section{Mathematical representation}

\pagebreak

\section{System Design}

\pagebreak

\documentclass{article}

\begin{document}

\section{Evaluation}

First of all, we needed to thoroughly understand the problem at hand. To achieve this, we organized a comprehensive discussion involving all team members. This discussion aimed to elucidate the various aspects of the problem, ensuring that everyone had a clear and vivid understanding of the issue. We explored different perspectives, asked clarifying questions, and shared relevant insights, all of which contributed to a more profound and collective grasp of the problem's intricacies. Then we divided into smaller groups so that we could work on frontend and backend at the same time.

\textbf{General Problems:}

\begin{enumerate}
    \item We encountered some difficulty in identifying all the edge cases of the problem due to the lack of clarity in certain descriptions.
    \item 
\end{enumerate}

\textbf{Frontend Problems:}

\begin{enumerate}
    \item Due to our initial unfamiliarity with the React library, we required a considerable amount of time to learn how to utilize it effectively to achieve the desired results.
\end{enumerate}

\textbf{Backend Problems:}

\begin{enumerate}
    \item 
\end{enumerate}

\end{document}

\pagebreak

\section{Problems and Conclusion}

\bibliographystyle{abbrv}
\bibliography{references}  % need to put bibtex references in references.bib 
\end{document}
